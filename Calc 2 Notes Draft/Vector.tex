\section{Vector}
I trust you took linear algebra before.

\begin{theorem}{}{}
Given a vector equation:
    \begin{equation}
        r(t)=f(t)\Vec{i}+g(t)\Vec{j}+h(t)\Vec{k}
    \end{equation}
Derivative of the equation is:
    \begin{equation}
        r'(t)=f'(t)\Vec{i}+g'(t)\Vec{j}+h'(t)\Vec{k}
    \end{equation}
Integral of the equation is:
    \begin{equation}
        \int r(t)=\int f(t)\Vec{i}+\int g(t)\Vec{j}+\int h(t)\Vec{k}+C
    \end{equation}
    C is any vector $<$a, b, c$>$ where a, b and c are all constants.
\end{theorem}

\subsection{Magnitude of Vector}
\begin{equation}
    ||r'(t)||=\sqrt{(f'(t))^2+(g'(t))^2+(h'(t))^2}
\end{equation}

\subsection{Unit vector}
We are interested in unit tangent vector:
\begin{equation}
    T(t)=\frac{r'(t)}{||r'(t)||}
\end{equation}

\subsection{Principal/Unit Normal Vector}
\begin{equation}
    N(t)=\frac{T'(t)}{||T'(t)||}
\end{equation}

\subsection{Binormal Vector}
\begin{equation}
    B(t)=T(t)\times N(t)
\end{equation}

\subsection{Arc Length}
\begin{equation}
    L=\int^b_a\sqrt{(f'(t))^2+(g'(t))^2+(h'(t))^2}dt=\int^b_a||r'(t)||dt
\end{equation}


\subsection{Arc Length Function}
Arc length function is basically expressing the function with distance traveled as the domain.
\newpage
\noindent Checklist:
\begin{enumerate}
    \item Find the distance traveled with respect to t
    \item Isolate for t
    \item Sub the equation back into r(t)
\end{enumerate}

\begin{simple}{}{}
Given 
$$r(t)=<2\sin{(t)}, 2\cos{(t)}>$$
find arc length function.

Solution:\\
Step 1:
\begin{align*}
    S&=\int^t_0||r'(\theta)||d\theta\\
    &=\int^t_0\sqrt{4\sin^2{(\theta)}+4\cos^2{(\theta)}} d\theta\\
    &=\int^t_0\sqrt{4}d\theta\\
    &=\int^t_02d\theta\\
    &=2t
\end{align*}

Step 2:
\begin{align*}
    S=&2t\\
    \frac{S}{2}=t
\end{align*}

Step 3:
$$r(t)=\left<2\sin{\left(\frac{S}{2}\right)}, 2\cos{\left(\frac{S}{2}\right)}\right>$$
\end{simple}
\noindent If starting at another point:\\
Checklist:
\begin{enumerate}
    \item Find $t_0$ at that point
    \item $\int^t_{t_0}||r'(t)||dt$
\end{enumerate}

\begin{simple}{}{}
Given the equation:
$$r(t)=<t^2+3, 2\cos{\pi t}>$$

find $t_0$ when $r(t)=<12, -2>$

Solution:
$t_0$=3
\end{simple}

\subsection{Curvature}
$$\kappa=\frac{||T'(t)||}{||r'(t)||}=\frac{||r'(t)\times r''(t)||}{||r'(t)||^3}$$

\subsection{Acceleration}
\begin{theorem}{}{}
Acceleration contains two components:
\begin{multicols}{2}
\noindent
$$a_T\text{ is tangential acceleration}$$
$$a_n\text{ is normal/perpendicular acceleration}$$
\end{multicols}
\begin{equation}
    \Vec{a}(t)=a_T\Vec{T}+a_n\Vec{N}
\end{equation}
Where $\Vec{T}$ is unit tangent vector and $\Vec{N}$ is unit normal vector.\\
Magnitude of tangential component is:
\begin{equation}
    a_T=\frac{r'(t)\cdot r''(t)}{||r'(t)||}
\end{equation}
Magnitude of normal component is:
\begin{equation}
    a_n=\frac{||r'(t)\times r''(t)||}{||r'(t)||}
\end{equation}
\end{theorem}

\begin{theorem}{}{}
If a cylinder can be expressed with equation $x^2+y^2=a$, think of it as a cylinder with a radius of $\frac{a}{2}$ centered at origin.
\end{theorem}

\begin{example}{}{}
Find the intersection of cylinder: $$x^2+y^2=a$$ and plane $$x+y+z=b$$
Solution:

We can create a variable t such that x and y value can be express in the form of $\frac{a}{2}\cos{(t)}$ and $\frac{a}{2}\sin{(t)}$ respectively.Taking these values back into plane equation:
\begin{align*}
    x+y+z&=b\\
    \frac{a}{2}\cos{(t)}+\frac{a}{2}\sin{(t)}+z&=b\\
    z&=b+\frac{a}{2}\cos{(t)}+\frac{a}{2}\sin{(t)}
\end{align*}

Therefore the intersection can be described by vector equation:
$$r(t)=\left<\frac{a}{2}\cos{(t)}, \frac{a}{2}\sin{(t)}, b+\frac{a}{2}\cos{(t)}+\frac{a}{2}\sin{(t)}\right>$$
\end{example}