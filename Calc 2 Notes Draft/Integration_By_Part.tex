\section{Integration By Part}
	\begin{equation}
	\int f(x)g'(x) dx = f(x)g(x)-\int f'(x)g(x)dx
	\end{equation}
	
	\begin{simple}{}{}
	Let try to integrate:
	$$\int(x+1)\cos{(x)}$$
	Let:
	
	\begin{multicols}{2}
	\noindent
	\begin{equation*}
	f(x)=x+1
	\end{equation*}
	\begin{equation*}
	g'(x)=cos(x)
	\end{equation*}
	\end{multicols}
	
	\noindent So that:
	
	\begin{multicols}{2}
	\noindent
	\begin{equation*}
	f'(x)=1
	\end{equation*}
	\begin{equation*}
	g(x)=\sin{(x)}
	\end{equation*}
	\end{multicols}
	
	\begin{align*}
	\int (x+1)\cos{(x)} &= (x+1)*\sin{(x)} -\int 1*\sin{(x)}\ dx\\
	&=(x+1)*\sin{(x)} - \int \sin{(x)}\ dx\\
	&=(x+1)*\sin{(x)} + \cos{(x)}\\
	\end{align*}
	\end{simple}
	
	\begin{simple}[colback=aliceblue, colframe=airforceblue]{}{}
	A weird case that you'll probably learn:
	\begin{align*}
	\int e^x\sin{(x)}\ dx
	\end{align*}
	Let:
	\begin{multicols}{2}
	\noindent
	\begin{equation*}
	f(x)=e^x
	\end{equation*}
	\begin{equation*}
	g'(x)=\sin (x)
	\end{equation*}
	\end{multicols}
	\noindent So that:
	\begin{multicols}{2}
	\noindent
	\begin{equation*}
	f'(x)=e^x
	\end{equation*}
	\begin{equation*}
	g(x)=-\cos (x)
	\end{equation*}
	\end{multicols}
	
	Sub everything into the formula:
	\begin{align*}
	\int e^x\sin{(x)}\ dx&=e^x*-cos (x)-\int e^x*-\cos (x)\ dx\\
	\int e^x\sin{(x)}\ dx&=-e^x * \cos (x)+\int e^x*cos (x)\ dx\\
	\end{align*}
	
	\newpage	
	
	Let:
	\begin{multicols}{2}
	\noindent
	\begin{equation*}
	f(x)=e^x
	\end{equation*}
	\begin{equation*}
	g'(x)=\cos (x)
	\end{equation*}
	\end{multicols}
	\noindent So that:
	\begin{multicols}{2}
	\noindent
	\begin{equation*}
	f'(x)=e^x
	\end{equation*}
	\begin{equation*}
	g(x)=\sin (x)
	\end{equation*}
	\end{multicols}
	
	\begin{align*}
	\int e^x\sin{(x)}\ dx&=-e^x*\cos{(x)}+\int e^x*cos(x)\ dx\\
	\int e^x\sin{(x)}\ dx&=-e^x*\cos{(x)}+\left[e^x*\sin{(x)}-\int e^x*\sin{(x)}\ dx\right]\\
	\int e^x\sin{(x)}\ dx&=-e^x*\cos{(x)}+e^x*\sin{(x)}-\int e^x*\sin{(x)}\ dx\\
	2\int e^x\sin{(x)}\ dx&=-e^x*\cos{(x)}+e^x*\sin{(x)}\\
	\int e^x\sin{(x)}\ dx&=-\frac{1}{2}e^x*\cos{(x)}+\frac{1}{2}e^x*\sin{(x)}
	\end{align*}
	\end{simple}
	
	\begin{theorem}{LIATE}{}
	Here's a quick checklist when picking f(x)
	\begin{enumerate}
	    \item L-Logarithmic 
	    \item I-Inverse Trigonometric
	    \item A-Algebra
	    \item T-Trigonometric
	    \item E-Exponential
	\end{enumerate}
	L is your most preferred f(x) and E is the worst choice.\footnotemark
	\end{theorem}
	\footnotetext{From my experience, you almost never have to pick L or I. T and E are exchangeable, pick the one that makes solving the question easier}